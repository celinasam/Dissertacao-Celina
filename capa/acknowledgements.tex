%\chapter*{Acknowledgements}
Agradeço minhas irmãs Eunice e Helena, que sempre me apoiaram nas minhas conquistas (foram muitas) e nos meus fracassos (foram poucos) e meus filhos Paula, Carolina e Daniel que entenderam minha ausência. Meus tios e tias, meus primos e primas também sempre me apoiando, querendo saber das novidades e perguntando: "quando você vem em Lins ?". A família é mesmo minha fortaleza.

Meus queridos amigos de quatro patas também tiveram muito a ver com esse trabalho. Uma parte da viabilização do trabalho foi obtida do faturamento das aulas de adestramento para cães e gatos que eu fiz entre 2005 e 2009. Depois disso, eu me afastei do trabalho com proprietários e continuei com o trabalho voluntário para as ONGs de posse responsável e castração, continuando a treinar animais resgatados das ruas para que eles ficassem mais adotáveis. A profa Maria de Fátima Martins (USP-FMVZ) e suas alunas Msc. Michele Ribeiro Silva e Msc. Juliana de Vazzi Pinheiro também foram muito importantes nesse processo difícil de voltar a academia. Ela me aconselhava em conversas: "Ela não tem que fazer outra faculdade!".

O colega de turma e amigo Paulo Cezar Campioni também apoiou meu trabalho de mestrado. Foi um pai sempre presente com nossos filhos.

Agradeço ao CNPq (Conselho Nacional de Desenvolvimento Científico e Tecnológico) pelo apoio concedido no início do meu mestrado e a FAPESP (Fundação de Amparo à Pesquisa do Estado de São Paulo) pelo apoio concedido ao meu projeto "Desenvolvimento e Qualidade de Dados para Sistema de Informação de Biodiversidade (Projeto de Pesquisa: Desenvolvimento de novas ferramentas e migração dos sistemas BIOTA do CRIA para o CENAPAD/UNICAMP)".

Agradeço aos professores, membros da banca. É uma honra contar com suas contribuições para meu trabalho.

O meu agradecimento também vai para os coordenadores de graduação e pós-graduação do IC, nesse período de 2010 a 2011, profo Ricardo da Silva Torres, profo Sandro Rigo, profo Hélio Pedrini, profo Alexandre Xavier Falcão e profo Paulo Lício de Geus.

Eu não queria falar muito do passado, mas eu tentei iniciar no programa de pós-graduação, logo que eu terminei a graduação em 1985 e uma outra vez, em 2002. Eu acredito que viram meu potencial, pois me aceitaram pela 3a vez no IC em 2010 e eu pude, então, concluir o programa de mestrado.

Já algum tempo, antes de 2008, quando eu quis aprender a teoria de computação que estava sendo oferecida nos cursos de graduação do IC, as páginas dos cursos do profo Rezende foram (e continuam a ser) importantes fontes de informação sobre essa área e me ajudaram muito a direcionar meus estudos. Obrigada, professor Rezende, por me aceitar como ouvinte nas disciplinas de Fundamentos Matemáticos da Computação (MC348) e Projeto e Análise de Algoritmos I (MC448), por proporcionar aulas tão bem elaboradas, lousas perfeitas e por demonstrar sempre disposição com as dúvidas que você me respondeu nos horários de atendimento. Fazer os quase 500 exercícios propostos nessas aulas me prepararam para as disciplinas da pós-graduação.

Também não posso me esquecer dos colegas de ciência da computação de 2007 que me adotaram como parte da turma, afinal, além das disciplinas com o profo Rezende, eu assisti Banco de Dados (MC536) com eles também! As animadas e surpreendentes aulas da profa Claudia também me prepararam para a pós. Obrigada Alex Grilo, Michael, Vanessa Schissato, Douglas Drummond, Bruno Malveira, Guilherme Paulovic, Guilherme Sampaio, Helen Her, Bráulio, "Panda" e demais colegas que eu lembro a fisionomia, mas não anotei o nome.

Agradeço também o profo Meidanis por me fazer estudar todos os dias nos meses de aulas de Complexidade de Algoritmos I (MO417). As técnicas utilizadas e o formato das aulas e das avaliações exigiram de mim mais participação na disciplina e diminuíram meu estresse com relação aos tópicos da disciplina. Foi uma boa convivência com meus colegas dessa turma também. Não teve turma mais legal!

Entre outras coisas, a profa Ariadne e o profo Buzato me ensinaram como resumir artigos, que é essencial para o desenvolvimento do trabalho da pós-graduação. Agradeço muito a eles por isso.

A oferta da disciplina Projeto e Implementação de Sistemas Distribuídos (MO641) pelo profo Buzato com trabalhos e seminários focados no projeto do Hadoop, também foi muito importante, pois pude conhecer e implementar aplicações que esse sistema distribuído de arquivos pode suportar.

Com o apoio da docente, a Profa Islene, minha orientadora, eu tive uma boa e interessante experiência para minha carreira de docente no estágio PED da disciplina Laboratório de Sistemas Distribuídos (MC715). Aprendi muito assistindo as apresentações dos alunos e instalando, compilando e testando o código java dos projetos dos alunos no Cluster do IC!

Obrigada Profa Islene, que sempre acreditou no meu potencial e me conduziu durante o mestrado. Pela sua atenção, paciência, experiência, tranquilidade, sorrisos, incentivo e críticas que contribuíram para o meu crescimento como aluna e como pessoa. Foi um grande privilégio tê-la como orientadora.

Não poderia deixar de agradecer o profo Tomasz pelas ótimas aulas que assisti na disciplina Estrutura de Dados (MC202) no programa de estágio PED e por ele sempre mostrar tanta qualidade e entusiasmo como professor. Responder as dúvidas no laboratório e por e-mail e fazer uma avaliação das tarefas dos alunos da turma de ciência da computação de 2011, com ajuda do colega Thiago Cavalcante, me estimularam a desenvolver técnicas para meu aprimoramento como docente.

Agradeço muito ao Rodrigo Schmidt do Facebook, ex-aluno do IC, pelas sugestões de temas, das quais escolhi \emph{erasure coding} e pela sua disposição em responder minhas dúvidas sobre a camada RAID do Hadoop. Não foi só ele que me ajudou nessa tarefa de conhecer o Hadoop. Também agradeço Scott Chen e Ramkumar Vadali.

Com a muito competente equipe da secretaria de pós-graduação, Daniel Capeleto, Fernando Okabe, Wilson Bagni Junior e Ademilson Ramos dos Reis, meu período como aluna no IC foi muito tranquilo. Pude contar com eles para todo apoio e para resolver qualquer problema.

Também quando um apoio de informática foi importante, Wiliam Lima Reiznautt e sua equipe sempre estavam dispostos a resolver a questão.

Sem música, alguns dias teriam sido muito difícies. Obrigada Paralamas do Sucesso, Skank, Legião Urbana, Capital Inicial, Tim Maia, Erasmo Carlos, Mutantes, Rita Lee, Celly Campello, Raul Seixas, Elis Regina, Nara Leão, João Gilberto, Chico Buarque, Paulinho da Viola, Carlos Lyra, Tom Jobim, Vinicius de Moraes, Tiê, Trio Irakitan, Almir Sater, Cascatinha e Inhana, Fafá de Belém, Marisa Monte, Arnaldo Antunes,  Adriana Calcanhoto, Renato Russo, Caetano Veloso, Gilberto Gil, Léo Jaime, Cazuza, Metrô, Premeditando o Breque, Lobão, Camisa de Vênus, Ira!, Roupa Nova, 14 Bis, Blitz, Kid Abelha, Morris Albert, Norah Jones, Aretha Franklin, B. J. Thomas, Nat King Cole, Elvis Presley, Chuck Berry, Jerry Lee Lewis, Roy Orbison, Frank Sinatra, The Platters, The Beatles, Suzi Quatro, Rolling Stones, Supertramp, Aerosmith, Nirvana, Led Zeppelin, Talking Heads, Sixpence None the Richer, Dire Straits, Aqualung, The Police, The Who, REM, Duran Duran, Peter Frampton, Art Garfunkel, John Lennon, Paul McCartney, George Harrison, Ringo Star, Paul Simon, David Bowie, Carlos Santana, Eric Clapton, Elton John, Carly Simon, Lynyrd Skynyrd, Beach Boys, Player, Three Dog Night, The Eagles, The Doors, Amy Winehouse, Bread, Carpenters, The Mamas and The Papas, Joan Baez, Lobo, The Monkees, Pink Floyd, The Alan Parsons Project, Nicolette Larson, Kate Bush, Laura Pausini, Dolly Parton, Earth, Wind and Fire, Michael Jackson e Avril Lavigne pelas letras e melodias e muitos outros artistas pelas músicas com arranjos para violão, gaita de fole e saxofone e com solos de guitarra.

Agradeço ainda toda comunidade de software livre por todas as informações que pude encontrar nos fóruns, nas listas de discussão, nos sites, nas documentações, por toda a conquista do software de qualidade obtida com o trabalho colaborativo, software esse testado por milhares de pessoas e em centenas de projetos e empresas. É nisso que eu acredito!



