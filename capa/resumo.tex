%\chapter*{Resumo}\label{ch:resumo}
  Os dados em um sistema distribuído confiável devem estar disponíveis
  quando for necessário. A codificação por apagamento (\emph{erasure
    codes}) tem sido utilizada por sistemas para alcançar requisitos
  de confiabilidade e de redução do custo de armazenamento de dados. O
  Hadoop é um \emph{framework} para execução de aplicações em
  armazenamento distribuído de grande volume de dados e que pode ser
  construído com \emph{commodity hardware}, que é facilmente acessível
  e disponível. Esta proposta apresentará uma análise da viabilidade
  da implementação prática de técnicas de codificação por apagamento
  no Hadoop \emph{Distributed File System} (HDFS), as alterações no
  Hadoop e a eficácia dessas alterações. Esta proposta é uma
  contribuição para \emph{software} livre em sistemas distribuídos.

