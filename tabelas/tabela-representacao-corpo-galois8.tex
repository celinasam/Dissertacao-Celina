\begin{table}
%\singlespacing
    \centerline{
    \begin{tabular}{ccc}\\ \hline
representa��o bin�ria & representa��o polinomial & representa��o exponencial\\ \hline
0000 & 0 & 0\\
1000 & $\alpha$ & $\alpha$\\
0100 & $\alpha^2$ & $\alpha^2$\\
0001 & $\alpha^3$ & $\alpha^3$\\
1100 & $\alpha + 1$ & $\alpha^4$\\
0110 & $\alpha + \alpha^2$ & $\alpha^5$\\
0011 & $\alpha^2 + \alpha^3$ & $\alpha^6$\\
1101 & $1 + \alpha + \alpha^3$ & $\alpha^7$\\
1010 & $1 + \alpha^2$ & $\alpha^8$\\
0101 & $\alpha + \alpha^3$ & $\alpha^9$\\
1110 & $\alpha^2 + \alpha^4$ & $\alpha^{10}$\\
0111 & $\alpha + \alpha^2 + \alpha^3$ & $\alpha^{11}$\\
1111 & $1 + \alpha + \alpha^2 + \alpha ^3$ & $\alpha^{12}$\\
1011 & $1 + \alpha^2 + \alpha^3$ & $\alpha^{13}$\\
1001 & $1 + \alpha^3$ & $\alpha^{14}$\\
    \end{tabular}}
    \caption{Mapeamento dos elementos do corpo $\mathbb{GF}_4$ gerado pelo polin�mio $1 + x^3 + x^4$}
    \label{tab7:comp}
\end{table}

