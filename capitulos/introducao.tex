\chapter{Introdução}

A codificação por apagamento (\emph{erasures codes}) introduz
redundância em um sistema de transmissão ou de armazenamento de dados de
maneira a permitir a detecção e correção de erros. A codificação por
apagamento é, desde os anos 70, utilizada pela \emph{NASA's Deep Space
  Network} para receber sinais e dados de telemetria
(\emph{downlinks}) vindos de veículos espaciais (\emph{very distant
  spacecrafts}) e para enviar telecomandos (\emph{uplinks}) para
veículos espaciais \cite{Almeida:2007, STO:2010, TDD:2010}.

A técnica de codificação por apagamento pode ser combinada com a
distribuição de dados entre vários dispositivos de armazenamento, o
que permite o aumento da largura de banda e a correção de
erros~\cite{Woitaszek:2007, Plank:1997}. Requisitos de confiabilidade
e de redução do tamanho do armazenamento podem ser observados em
sistemas que tratam de: 
%\emph{digital fountain} (\emph{multicasting} multimídia confiável)\cite{Byers:1998}; 
\emph{Delay and Disruption Tolerant Networks}, redes de sensores e
redes~\emph{peer-to-peer} \cite{Bhagwan:2004, Haeberlen:2005,
Rodrigues:2005, RTAD:2007, Wilcox-O'Hearn:2008, Houri:2009} e
armazenamento de grande volume de dados \cite{Anderson:1998,
Kubiatowicz:2000, Schmuck:2002, Saito:2004, Xia:2006, Storer:2008,
Storer:2009}, como também o sistema de arquivos distribuído do
Hadoop (HDFS)~\cite{HDFS-503:2010}.

O HDFS, por padrão, implementa alta disponibilidade dos dados via
replicação simples dos blocos de dados. Esta abordagem acarreta um
alto custo de armazenamento para garantir que os dados estarão sempre
disponíveis. O objetivo do uso da codificação por apagamento no HDFS é
permitir que o espaço de armazenamento possa ser reduzido sem
prejudicar a disponibilidade dos dados. Esforços iniciais nessa linha
foram feitos utilizando técnicas de RAID~\cite{HDFS-503:2010} e mais
recentemente do algoritmo Reed-Solomon~\cite{MR-1969:2010}.

Este trabalho pretende avançar esta linha de pesquisa a partir dos
seguintes passos:

\begin{itemize}
\item avaliação de desempenho, ganhos, e custos de diferentes
  estratégias de codificação por apagamento;

\item implementação de otimizações ou extensões para o código que
  atualmente implementa Reed-Solomon, tentando melhorar,
  principalmente, a parte de distribuição de blocos;

\item implementação de novos algoritmos (e.g., Tornado codes) e
  extenção da interface atual para aceitá-los;

\item integração do código atual com o HDFS.

\end{itemize}

O texto a seguir está organizado da seguinte maneira: os Capítulos 2 e 3
introduzem os conceitos básicos da álgebra abstrata e da codificação por apagamento, respectivamente, o Capítulo 4 apresenta uma discussão sobre esquemas de redundância de dados, o Capítulo 5 comenta o \emph{framework} Hadoop e seu sistema de arquivos, o Capítulo 6 comenta as codificações Tornado e Turbo-\emph{Like} que foram implementadas para uma versão do Hadoop e o Capítulo 7 apresenta as contribuições e conclusões deste trabalho.

