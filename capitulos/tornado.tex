\chapter{Codifica��o Tornado}

A id�ia b�sica do algoritmo da codifica��o Tornado est� descrita em \cite{Stoten:2011}.

Um checksum de 64 bits pode ser calculado para cada 100 bytes.

Para um bloco de 64MB, o checksum � de 5.12MB.

sobrecarga de checksum: $864/800 = 1.08$

sobrecarga de paridade: $2n/n = 2$

sobrecarga total: $1.08*2 = 2.16$

l = n�mero de falhas suportadas = n�mero de blocos da stripe = 10

\subsection{Algoritmo de Codifica��o}

Exemplo do algoritmo de codifica��o

O tamanho da \emph{stripe} � 10 blocos e existe um arquivo $/a/arquivo.txt$ com exatamente 10 blocos. Nesse caso, o algoritmo de codifica��o faz o seguinte:

\begin{itemize}
\item bloco$[0]\ =$ primeiro bloco da stripe
\item bloco$[1]\ =$ segundo bloco da stripe
\item ...
\item bloco$[9]\ =$ �ltimo bloco da stripe

\item para i de 0 at� 9
	\begin{itemize}
		\item bloco\_checksum$[i]\ =\ $checksum do bloco$[i]$
	\end{itemize}

\item bloco\_paridade$[0]\ =\ $bloco$[0]\ xor\ $bloco$[1]$
\item bloco\_paridade$[1]\ =\ $bloco$[2]\ xor\ $bloco$[3]$
\item bloco\_paridade$[2]\ =\ $bloco$[4]\ xor\ $bloco$[5]$
\item bloco\_paridade$[3]\ =\ $bloco$[6]\ xor\ $bloco$[7]$
\item bloco\_paridade$[4]\ =\ $bloco$[8]\ xor\ $bloco$[9]$
\item bloco\_paridade$[5]\ =\ $bloco$[0]$
\item bloco\_paridade$[6]\ =\ $bloco$[2]$
\item bloco\_paridade$[7]\ =\ $bloco$[4]$
\item bloco\_paridade$[8]\ =\ $bloco$[6]$
\item bloco\_paridade$[9]\ =\ $bloco$[8]$

\item para i de 0 at� 9:
      \begin{itemize}
    \item escreva bloco\_checksum$[i]$ no arquivo $/checksum/a/arquivo.txt$
      \end{itemize}

\item para i de 0 at� 9:
      \begin{itemize}
    \item escreva bloco\_paridade$[i]$ no arquivo $/tornado/a/arquivo.txt$
      \end{itemize}
\end{itemize}

\subsection{Algoritmo de Decodifica��o}

Exemplo do algoritmo de decodifica��o

O tamanho da \emph{stripe} � 10 blocos e existe um arquivo $/a/arquivo.txt$ com exatamente 10 blocos. Nesse caso, o algoritmo de decodifica��o faz o seguinte:

\begin{itemize}

\item bloco\_checksum$[0]\ =$ primeiro bloco de checksum da stripe
\item bloco\_checksum$[1]\ =$ segundo bloco de checksum da stripe
\item ...
\item bloco\_checksum$[9]\ =$ �ltimo bloco de checksum da stripe

\item $i\ =\ 0$
\item $erro\ =\ false$
\item enquanto $(i\ <=\ 9)\ e\ (nao\ erro)$ fa�a
	\begin{itemize}
		\item	se bloco\_checksum$[i]\ =\ $checksum do bloco$[i]$
		\item	ent�o $i++$
		\item	sen�o $erro\ =\ true$
	\end{itemize}

\item se erro
\item ent�o
	\begin{itemize}
		\item bloco$[0]\ =\ $bloco\_paridade$[5]$
		\item bloco$[2]\ =\ $bloco\_paridade$[6]$
		\item bloco$[4]\ =\ $bloco\_paridade$[7]$
		\item bloco$[6]\ =\ $bloco\_paridade$[8]$
		\item bloco$[8]\ =\ $bloco\_paridade$[9]$
		\item bloco$[1]\ =\ $bloco\_paridade$[0] xor $bloco$[0]$
		\item bloco$[3]\ =\ $bloco$[2]\ xor\ $bloco\_paridade$[1]$
		\item bloco$[5]\ =\ $bloco$[4]\ xor\ $bloco\_paridade$[2]$
		\item bloco$[7]\ =\ $bloco$[6]\ xor\ $bloco\_paridade$[3]$
		\item bloco$[9]\ =\ $bloco$[8]\ xor\ $bloco\_paridade$[4]$
	\end{itemize}
\end{itemize}
