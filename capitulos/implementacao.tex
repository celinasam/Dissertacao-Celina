\chapter{Implementação das Codificações}

Uma boa escolha de codificação para um canal depende da sobrecarga de armazenamento, do tamanho do bloco, da complexidade dos algoritmos de codificação e de decodificação e se uma eventual não detecção de erros é aceitável. Códigos LDPC e Tornado são versáteis para serem projetados para muitas opções de tamanho do bloco e de sobrecarga de armazenamento e segundo muitos estudos, códigos LDPC são a codificação que mais se aproxima do limite de Shannon (quantidade máxima de dados transmitidos em um canal em bytes por segundo).

\section{Codificação Tornado}

A idéia básica deste algoritmo da codificação Tornado está descrita em \cite{Woitaszek:2007,Stoten:2011}.

O cálculo do checksum de 64 bits foi feito para cada 100 bytes. Para um bloco de 64MB, o checksum é de 5.12MB.

A sobrecarga de armazenamento de checksum é $\frac{864}{800} = 1.08$. A sobrecarga da paridade é $\frac{2n}{n} = 2$. Portanto, a  sobrecarga total de armazenamento é $1.08*2 = 2.16$.

O número de falhas suportadas é $l\ =\ numero\ de\ blocos\ da\ stripe\ =\ s$.

A operação mais custosa dos algoritmos é o cálculo do $XOR$ entre 2 blocos de 64M. São $\frac{s}{2}$ operações, no caso de $s$ par e $\frac{n}{2}-1$ operações, no caso de $s$ ímpar. Nesse caso, podemos afirmar que os algoritmos são lineares no tamanho da entrada, ou seja, $O(s)$.

O grafo da codificação para uma \emph{stripe} de tamanho 10 blocos é:

\begin{verbatim}
1 0 0 0 0 0 0 0 0 0
1 1 0 0 0 0 0 0 0 0
0 1 1 0 0 0 0 0 0 0
0 0 1 1 0 0 0 0 0 0
0 0 0 1 1 0 0 0 0 0
0 0 0 0 1 1 0 0 0 0
0 0 0 0 0 1 1 0 0 0
0 0 0 0 0 0 1 1 0 0
0 0 0 0 0 0 0 1 1 0
0 0 0 0 0 0 0 0 1 1
\end{verbatim}


\subsection{Algoritmo de Codificação}

Apresentamos o algoritmo de codificação para uma \emph{stripe} de tamanho $s$ blocos e que gera $s$ blocos de paridade. 

\begin{verbatim}
// Ler o arquivo /a/arquivo.txt
para i de 0 até s-1:
   bloco[i] = i-ésimo bloco da stripe

para i de 0 até s-1:
   se i = 0
   então
      bloco_paridade[i] = bloco[i]
   senão
      bloco_paridade[i] = bloco[i-1] xor bloco[i]
   i++

para i de 0 até s-1:
   escreva bloco_paridade[i] no arquivo /tor/a/arquivo.txt
\end{verbatim}

\subsection{Algoritmo de Decodificação}

O algoritmo de decodificação faz o seguinte:

\begin{verbatim}
// Ler  o arquivo  /tor/a/arquivo.txt
para i de 0 até s-1:
   bloco_paridade[i] = i-ésimo bloco de paridade da stripe

para i de 0 até s-1:
   se i = 0
   então
      bloco[i] = bloco_paridade[i]
   senão
      bloco$[i] = bloco_paridade[i] xor bloco[i-1]
   i++
\end{verbatim}

\section{Codificação Simples Turbo-\emph{Like}}

Esse algoritmo foi baseado nos estudos de \cite{Divsalar:1998,MacKay:2003} e é muito simples de entender. 

A sobrecarga total de armazenamento é $\frac{2n}{n}=2$.

O número de falhas suportadas é $l\ =\ numero\ de\ blocos\ da\ stripe\ =\ s$.

A operação mais custosa dos algoritmos é o cálculo do $XOR$ entre 2 blocos de 64M. São $s$ operações. Nesse caso, podemos afirmar que os algoritmos são lineares no tamanho da entrada, ou seja, $O(s)$.

O grafo da codificação para uma \emph{stripe} de tamanho 10 blocos na sequência $7, 1, 4, 8, 2, 5, 9, 3, 0, 6$:

\begin{verbatim}
0 0 0 0 0 0 0 1 0 0
0 1 0 0 0 0 0 1 0 0
0 1 0 0 1 0 0 0 0 0
0 0 0 0 1 0 0 0 1 0
0 0 1 0 0 0 0 0 1 0
0 0 1 0 0 1 0 0 0 0
0 0 0 0 0 1 0 0 0 1
0 0 0 1 0 0 0 0 0 1
1 0 0 1 0 0 0 0 0 0
1 0 0 0 0 0 1 0 0 0
\end{verbatim}


\subsection{Algoritmo de Codificação}

Apresentamos o algoritmo de codificação para uma \emph{stripe} de tamanho $s$ blocos e que gera $s$ blocos de paridade.

\begin{verbatim}
// Ler o arquivo /a/arquivo.txt
para i de 0 até s-1:
   bloco[i] = i-ésimo bloco da stripe

Gerar uma permutação dos índices dos blocos da stripe: p = [b_0,
b_1, b_2, b_3, ..., b_{s-2}, b_{s-1}]

para i de 0 até s-1:
   se i = 0
   então
      bloco_paridade[i] = bloco[p[i]]
   senão
      bloco_paridade[i] = bloco_paridade[i-1] xor bloco[p[i]]
   i++

para i de 0 até s-1:
   escreva bloco_paridade[i] no arquivo /turbo/a/arquivo.txt
\end{verbatim}

\subsection{Algoritmo de Decodificação}

O algoritmo de decodificação faz o seguinte:

\begin{verbatim}
Ler a permutação do índices dos blocos da stripe: p =  [b_0,
b_1, b_2, b_3, ..., b_{s-2}, b_{s-1}]

// Ler  o arquivo  /turbo/a/arquivo.txt
para i de 0 até s-1:
   bloco_paridade[i] = i-ésimo bloco de paridade da stripe

para i de 0 até s-1:
   se i = 0
   então
      bloco[p[i]] = bloco_paridade[i]
   senão
      bloco[p[i]] = bloco_paridade[i-1] xor bloco_paridade[i]
   i++
\end{verbatim}

\section{Comparação entre as Codificações}
\begin{table}
%\singlespacing
    \centerline{
    \begin{tabular}{|p{3cm}|p{1.5cm}|p{1.5cm}|p{1.2cm}|p{2cm}|p{1.7cm}}\hline
	{\bf Codifica��o} & {\bf N�mero de falhas suportadas} &  {\bf Espa�o de armazenamento} & {\bf Grau de reparacao} & {\bf Corrup��o de um dado} & {\bf Complexidade dos algoritmos}\\ \hline
	Replica��o 2X & $l=2$ & $B=2$ & $d=1$ & $e = q^2$ & $O(1)$\\ \hline
	RAID (6, 5) & $l=1$ & $B=1.2$ & $d=5$ & $e = q$ & $O(5log 6)$\\ \hline
	RS (8, 5) & $l=3$ & $B=1.6$ & $d=5$ & $e = q^3$ & $O(5(3 log 8))$\\ \hline
	Tornado (10, 5) & $l=5$ & $B=2$ & $d=5$ & $e = q^5$ & $O(10)$\\ \hline
	Turbo-\emph{Like} (10, 5) & $l=5$ & $B=2$ & $d=5$ & $e = q^5$ & $O(10)$\\ \hline
	Replica��o $nX$ & $l=n$ & $B=n$ & $d=1$ & $e = q^{n-1}$ & $O(n)$\\ \hline
	RS $(2k, k) ou (2k-1,k)$ & $l=k$ & $B=2$ & $d=k$ & $e = q^k$ & $O(k^2log 2k) \cite{Luby:2002}$\\ \hline
	Tornado $(2k, k) ou (2k-1,k)$ & $l=k$ & $B=2$ & $d=k$ & $e = q^k$ & $O(2k) \cite{Luby:2002}$\\ \hline
	Turbo-\emph{Like} $(2k, k) ou (2k-1, k)$ & $l=k$ & $B=2$ & $d=k$ & $e = q^k$ & $O(2k)$\\ \hline
    \end{tabular}}
    \caption{Compara��o entre as codifica��es implementadas (nem todas ainda!) no HDFS}
    \label{tab2:comp}
\end{table}


